\documentclass{article}
\usepackage[utf8]{inputenc}
\usepackage[final]{nips_2018}

\title{Generative models and the manifold hypothesis}
\author{Alice I.~Cecile \\
        Dreaming Machines Corporation \\
        Kitchener, Ontario, Canada \\
        \texttt{alice.i.cecile@gmail.com}}
\date{December 8, 2018}

\begin{document}

\maketitle

\begin{abstract}
    Generative models are powerful tools for creating art, but existing techniques to interact with them are ad hoc and unprincipled.
    We present a unifying conceptual framework for their design and manipulation, based on the hypothesis that the true data distribution lies on a manifold.
    Generative models which have an explicit latent space (such as GANs and autoencoders) all attempt to learn a homeomorphism between the latent space and the data manifold.
    As a result, we can conclude that the topology and dimensionality of the latent space are its most salient features.
    Furthermore, this perspective allows us to define a common set of principled methods for interaction with generative models, opening the door to a direct reusable interface for artists.
\end{abstract}

\section{Introduction}

THE MANIFOLD HYPOTHESIS.

LATENT SPACE MODELS.

OBSERVATIONS OF MEANINGFUL INTERPOLATION.

\section{The topology of concepts}

WHY WE MIGHT EXPECT TO SEE DIFFERENT TOPOLOGIES.

WHY TOPOLOGY AND DIMENSION ARE THE ONLY THINGS THAT MATTER.

DESIDERATA: SAMPLING AND INTERPOLATION

SOME TRACTABLE TOPOLOGIES

\section{Manifold-driven interaction}

CONVERSION.

VECTOR MATH.

INTERPOLATION.

\section{Conclusion}

THEORETICAL INSIGHTS INTO DESIGN.

PRESENTS A COMMON INTERFACE TO TARGET.

%\makebibliography

\end{document}
